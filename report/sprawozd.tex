%@descr: wzór sprawozdania, raportu lub pracy - nadaje się do przeróbek
%@author: Maciej Komosiński

\documentclass{article} 
\usepackage{polski} %moze wymagac dokonfigurowania latexa, ale jest lepszy niż standardowy babel'owy [polish] 
\usepackage[utf8]{inputenc} 
\usepackage[OT4]{fontenc} 
%\usepackage{gensymb}
\usepackage{graphicx,color} %include pdf's (and png's for raster graphics... avoid raster graphics!) 
\usepackage{url} 
\usepackage[pdftex,hyperfootnotes=false,pdfborder={0 0 0},colorlinks=true, linkcolor=blue]{hyperref} %za wszystkimi pakietami; pdfborder nie wszedzie tak samo zaimplementowane bo specyfikacja nieprecyzyjna; pod miktex'em po prostu nie widac wtedy ramek


\input{_ustawienia.tex}

%\title{Sprawozdanie z laboratorium:\\Metaheurystyki i Obliczenia Inspirowane Biologicznie}
%\author{}
%\date{}


\begin{document}

\thispagestyle{empty} %bez numeru strony

\begin{center}
{\large{Sprawozdanie z laboratorium:\\
Informatyka w Medycynie\\
(szablon)}}

\vspace{3ex}

Część I: Symulator tomografu komputerowego
%Część II: Algorytmy optymalizacji lokalnej i globalnej, problem QAP
%Część III: Eksperyment: ... (prezentację można zrobić w LaTeX - służy do tego klasa "beamer")

\vspace{3ex}
{\footnotesize\today}

\end{center}


\vspace{10ex}

Prowadzący: mrg inż. Iwo Błądek

\vspace{5ex}

Autorzy:
\begin{tabular}{lllr}
\textbf{Sebastian Firlik} & inf122485 & I2 & sebastian.firlik@student.put.poznan.pl \\
\textbf{Piotr Hankiewicz} & inf1225** & I2 & MAIL \\
\end{tabular}

\vspace{5ex}

Zajęcia piątkowe, 11:45.

\vspace{35ex}

\noindent Oświadczam/y, że niniejsze sprawozdanie zostało przygotowane wyłącznie przez powyższych autora/ów,
a wszystkie elementy pochodzące z innych źródeł zostały odpowiednio zaznaczone i~są cytowane w bibliografii.  

\newpage



\section*{Udział autorów}
\begin{tightlist}
\item SF zaimplementował tworzenie sinogramu, a także jego zapis do formatu DICOM, przygotował środowisko graficzne aplikacji
\item PH zaimplementował odtwarzanie obrazu wejściowego na podstawie sinogramu wraz z filtrowaniem, przeprowadził eksperyment wyjaśniający wpływ niżej opisanych parametrów na jakoś odtwarzanego obrazu.
\end{tightlist}



\section{Wstęp}

Naszym zadaniem było stworzenie aplikacji desktopowej w wybranej technologii i zaimplementowanie w niej symulacji dwuwymiarowego tomografu komputerowego. Wszystkie wymagania zostały wypunktowane \href{https://www.cs.put.poznan.pl/ibladek/students/iwm/0_projekt_wspolny_Tomograf.pdf}{tutaj}. Do stworzenia naszej aplikacji użyliśmy:
\begin{tightlist}
\item języka Python 3,
\item środowiska PyQt5 do stworzenia interfejsu okienkowego,
\item bibliotek dostępnych w języku Python (matplotlib, numpy, pyDicom...)
\end{tightlist}

Jedynie do obsługi zapisu do formatu DICOM użyliśmy gotowej biblioteki ze względu na trudność manipulacji danymi w tym formacie. Wszystkie obliczenia, zarówno podczas generacji sinogramu, jak i przejściu do obrazu wynikowego, a także obrót emitera i detektorów w funkcji kąta zamodelowaliśmy samodzielnie.

\begin{figure}[!htbp]
\begin{center}
\includegraphics[width=0.8\textwidth]{tomograf.jpg}
\end{center}
\caption{Schemat działania tomografu}
\label{fig:1Tdelta}
\end{figure}

Na Rysunku 1 widzimy, jak działa nasz tomograf komputerowy - jeden emiter obraca się o $360 ^{\circ}$, wysyła promieniowanie rentgenowskie przez badany obiekt i każda wiązka trafia na odpowiedni detektor. W naszej symulacji, przy pomocy algorytmu Bresenhama tworzymy dyskretną linię, prowadzącą od emitera do detektora przez obraz i sumuje jasności pikseli na tej drodze. W ten sposób, na każdej ścieżce emiter-detektor, dla każdego możliwego kąta obrotu emitera, otrzymujemy \textbf{sinogram}, czyli pośredni etap wizualizacji badanego obiektu. Nie podlega on żadnej analizie diagnostycznej. Następnie w wyniku operacji odwrotnej (Odwrotna Dyskretna Transformata Radona) otrzymujemy obraz wynikowy. 

ODTR polega na tym, że z każdym kątem obrotu $\alpha$ i każdą parą emiter-detektor sprzężona jest suma jasności pikseli na ich drodze. Na każdym pikselu, znajdującym się na tej drodze, zostawiamy średnią jas odpowiednieność piksela, wynikającą zj sumy z sinogramu. Po iteracyjnym odtworzeniu obrazu, bez nałożonego filtrowania widzimy niedokładny, rozmyty obraz, podobny do początkowego. NORMALIZACJA? 

RYSUNEK-PIERWOTNY, SINOGRAM, ODTWORZONY

\clearpage %pozwol umiescic zalegle rysunki od razu tutaj 


\section{Analiza statystyczna}
\label{sec_analiza_statystyczna}

\subsection{Wpływ kąta $\alpha$ na jakość przetwarzania}
\label{subsec_alpha_comparison}

Pierwszym z parametrów wpływających na jakość przetwarzania jest kąt o jaki obraca się emiter w jednej iteracji. 

\begin{figure}[!htbp]
\begin{center}
\includegraphics[width=0.8\textwidth]{alpha.pdf}
\end{center}
\caption{Wpływ kąta alfa na jakość przetwarzania}
\label{fig:detectors_amount}
\end{figure}


\subsection{Wpływ liczby detektorów na jakość przetwarzania}
\label{subsec_detectors_amount_comparison}

Kolejnym z czynników jest liczba detektorów. Na poniższym zestawieniu widać wyraźnie, że wraz z rosnącą liczbą detektorów jakość przetwarzania rośnie. Na osi y przedstawione są wartości błędu średniokwadratowego dla kolejnych wartości liczb detektorów, które znajdują się na osi x. 

\begin{figure}[!htbp]
\begin{center}
\includegraphics[width=0.8\textwidth]{detectors_amount.pdf}
\end{center}
\caption{Wpływ liczby detektorów na jakość przetwarzania}
\label{fig:detectors_amount}
\end{figure}

\subsection{Wpływ rozpiętości kątowej detektorów na jakość przetwarzania}
\label{subsec_cone_width_comparison}


\begin{figure}[!htbp]
\begin{center}
\includegraphics[width=0.8\textwidth]{cone_width.pdf}
\end{center}
\caption{Wpływ rozpiętości kątowej na jakość przetwarzania}
\label{fig:detectors_amount}
\end{figure}

\section{Filtr}

%%%%%%%%%%%%%%%% literatura %%%%%%%%%%%%%%%%

\bibliography{sprawozd}
\bibliographystyle{plain}


\end{document}

